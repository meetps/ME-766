%%% Template originaly created by Karol Kozioł (mail@karol-koziol.net) and modified for ShareLaTeX use

\documentclass[a4paper,11pt]{article}

\usepackage[T1]{fontenc}
\usepackage[utf8]{inputenc}
\usepackage{graphicx}
\usepackage{xcolor}

\renewcommand\familydefault{\sfdefault}
\usepackage{tgheros}
\usepackage[defaultmono]{droidmono}

\usepackage{amsmath,amssymb,amsthm,textcomp}
\usepackage{enumerate}
\usepackage{multicol}
\usepackage{tikz}

\usepackage{geometry}
\geometry{total={210mm,297mm},
left=25mm,right=25mm,%
bindingoffset=0mm, top=20mm,bottom=20mm}


\linespread{1.3}

\newcommand{\linia}{\rule{\linewidth}{0.5pt}}

% custom theorems if needed
\newtheoremstyle{mytheor}
    {1ex}{1ex}{\normalfont}{0pt}{\scshape}{.}{1ex}
    {{\thmname{#1 }}{\thmnumber{#2}}{\thmnote{ (#3)}}}

\theoremstyle{mytheor}
\newtheorem{defi}{Definition}

% my own titles
\makeatletter
\renewcommand{\maketitle}{
\begin{center}
\vspace{2ex}
{\huge \textsc{\@title}}
\vspace{1ex}
\\
\linia\\
\@author \hfill \@date
\vspace{4ex}
\end{center}
}
\makeatother
%%%

% custom footers and headers
\usepackage{fancyhdr}
\pagestyle{fancy}
\lhead{}
\chead{}
\rhead{}
\lfoot{Assignment \textnumero{} 1}
\cfoot{}
\rfoot{Page \thepage}
\renewcommand{\headrulewidth}{0pt}
\renewcommand{\footrulewidth}{0pt}
%

% code listing settings
\usepackage{listings}
\lstset{
    language=Python,
    basicstyle=\ttfamily\small,
    aboveskip={1.0\baselineskip},
    belowskip={1.0\baselineskip},
    columns=fixed,
    extendedchars=true,
    breaklines=true,
    tabsize=4,
    prebreak=\raisebox{0ex}[0ex][0ex]{\ensuremath{\hookleftarrow}},
    frame=lines,
    showtabs=false,
    showspaces=false,
    showstringspaces=false,
    keywordstyle=\color[rgb]{0.627,0.126,0.941},
    commentstyle=\color[rgb]{0.133,0.545,0.133},
    stringstyle=\color[rgb]{01,0,0},
    numbers=left,
    numberstyle=\small,
    stepnumber=1,
    numbersep=10pt,
    captionpos=t,
    escapeinside={\%*}{*)}
}

%%%----------%%%----------%%%----------%%%----------%%%

\begin{document}

\title{Programming Assignment \textnumero{} 1}

\author{\textbf{Meet Pragnesh Shah | 13D070003 | ME 766}}

\date{23/01/2016}

\maketitle

\section{Problem 1}

\subsection{(a)}

We are given that the serial portion $S$ of the algorithm constitutes of $20$ \% of the total computations. Let's assume the number of processors to be $P$ and the parallelizable part of the code to be $C$. Thus we can say that the 

\begin{align*}
Speedup ( \gamma )   &= \frac{T_1}{T_p} \\
&= \frac{1}{S + \frac{C}{P}}
\end{align*}

Now we need to find the maximum possible speedup i.e the speedup when number of processes are infinte. It is hence equal to 

\begin{align*}
Max. Speedup &= \lim_{P\to\infty} \gamma \\
&\approx \frac{1}{S} \\
&\approx \textbf{5}
\end{align*}

\subsection{(b)}

We need to find the serial part of the computation in the algorithm given the desirable maximum speedup $\gamma$ as \textbf{50}.

Hence using the above formula of maximum achieveable speedup we get : \begin{align*}
S &=  \gamma^{-1} \\
&\approx \frac{1}{50} \\
&\approx 0.02 \\
&\approx \textbf{2 \%}
\end{align*}
	


\vspace*{2cm}
\section{Problem 2}

\subsection{(a)}

We are given the following specifications for the hypothetical Von Neumann Machine : 

\begin{align*}
Cache Size &=  1024 bytes \\
Memory Size &= 10 * 1024 * 1024 bytes  \\
Cache to Memory Block Size &=  1024 bytes\\
Fetch Time (from Memory) &= 150 CPU cycles \\
Fetch Time (from cache) &= 1 CPU cycles 
\end{align*}

We are given a $2D$ array \textbf{A[i][j]} of size $1024 *1024$ occupied by \textbf{ints} each of which needs 4 bytes assuming standard \textbf{int\_32}. 

Thus total memory in bytes we need to access from the main memory is : 
\begin{align*}
Size(A) = 1024 * 1024 * 4 bytes 
\end{align*}

Since we have a cache that can have a block of size $1024$ bytes at any instance , we can assume that when the loop executes from the beginning and tries to access the element \textbf{A[0][0]} a block of memory containing elements \textbf{A[0][0]} to \textbf{A[0][256]} is copied to the cache from the main memory as the block size of the above data is $256*4bytes$ = $Cache Size$ = $CachetoMemoryBlockSize$. This assumption only holds true since \textbf{A} is an array and by the nature of it the data is stored in continuous memory registers unlike other complex data structures.

Thus the cost to transfer the entire array \textbf{A} is :

\begin{align*}
Time Cost &= \frac{Size(A)}{Cache to Memory Block Size} * [150 + \frac{CacheSize}{Size(int)}*1 ] CPU cycles\\
&= 1662976 CPU cycles
\end{align*}


\subsection{(b)}

The difference between \textbf{FORTRAN} and \textbf{C} is that the former stores data in a \textbf{column major} manner and the latter uses \textbf{row major} system to store arrays in memory. Since the cache retrieves a block of data from the memory and the computation in the given code is done row-wise, similar code in \textbf{FORTRAN} will perform poorly as the number of cache misses will be very high and the blocks will have to be copied from memory to cache for every element in the 2D array. For each element $A_{i,j}$ the cache contains elements in the same column upto $A_{i+256,j}$ , but unfortunately they are of no use to us as the computation is row wise. Thus the cost to transfer the entire array \textbf{A} in \textbf{FORTRAN} will be :

\begin{align*}
Time Cost &= \frac{Size(A)}{Size(int)}* [Fetch Time (from Memory)  + Fetch Time (from Cache)]\\
&= \frac{1024 * 1024 * 4}{4} * [150 + 1] \\
&=  158334976 CPU cycles
\end{align*}

% % code from http://rosettacode.org/wiki/Fibonacci_sequence#Python
% \begin{lstlisting}[label={list:first},caption=Sample Python code -- Fibonacci sequence calculated analytically.]
% from math import *

% # define function 
% def analytic_fibonacci(n):
%   sqrt_5 = sqrt(5);
%   p = (1 + sqrt_5) / 2;
%   q = 1/p;
%   return int( (p**n + q**n) / sqrt_5 + 0.5 )
 
% # define range
% for i in range(1,31):
%   print analytic_fibonacci(i)
% \end{lstlisting}

% Following Listing~\ref{list:first}\ldots{} 
% Lorem ipsum dolor sit amet, consectetur adipiscing elit, sed do eiusmod tempor incididunt ut labore et dolore magna aliqua. Ut enim ad minim veniam, quis nostrud exercitation ullamco laboris nisi ut aliquip ex ea commodo consequat. Duis aute irure dolor in reprehenderit in voluptate velit esse cillum dolore eu fugiat nulla pariatur. Excepteur sint occaecat cupidatat non proident, sunt in culpa qui officia deserunt mollit anim id est laborum.

% \section*{Problem 2}

% \begin{lstlisting}[label={list:second},caption=Sample Bash code.]
% #! /bin/bash
% python stage1.py
% echo "Stage I done!"
% python stage2.py
% echo "Stage II done!"
% python stage3.py
% echo "Stage III done!"
% \end{lstlisting}

% Lorem ipsum dolor sit amet, consectetur adipiscing elit, sed do eiusmod tempor incididunt ut labore et dolore magna aliqua. Ut enim ad minim veniam, quis nostrud exercitation ullamco laboris nisi ut aliquip ex ea commodo consequat. Duis aute irure dolor in reprehenderit in voluptate velit esse cillum dolore eu fugiat nulla pariatur. Excepteur sint occaecat cupidatat non proident, sunt in culpa qui officia deserunt mollit anim id est laborum.

\end{document}
